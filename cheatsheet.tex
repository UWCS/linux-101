\documentclass[a4paper,11pt,parskip=half-]{scrartcl}

% Packages

\usepackage[english]{babel}
\usepackage{array}
\usepackage{amsmath}
\usepackage{ascii}
\usepackage[colorlinks=true, allcolors=blue]{hyperref}
\usepackage{graphicx}
\usepackage[a4paper,top=2cm,bottom=2cm,left=2cm,right=2cm,marginparwidth=1.75cm]{geometry}
\usepackage[defaultfam,tabular,lining]{montserrat}
\usepackage[T1]{fontenc}
\usepackage{wrapfig}


\graphicspath{{images/}}
\pagestyle{empty}
\setkomafont{section}{\usefont{T1}{fvs}{b}{n}\Large}
\renewcommand*\oldstylenums[1]{{\fontfamily{Montserrat-TOsF}\selectfont #1}}
\renewcommand{\ttdefault}{pcr}
\renewcommand{\arraystretch}{1.5}


\begin{document}

\begin{wrapfigure}{r}{0.2\textwidth}
    \vspace{-10pt} % Move shield in line with text
    \includegraphics[width=0.9\linewidth]{shield.png}
    \vspace{-100pt} % Prevent text below from moving out of the way
\end{wrapfigure}

\normalfont \Huge \bfseries UWCS Linux 101 Cheatsheet

\normalfont\LARGE Programs, Commands \& Tips
\normalsize

\section*{Programs}

You usually just need to type the name of a program into the terminal to run it, with a few exceptions. You can usually pass a filepath as an argument too. Here are some examples:

\centering
\begin{tabular}{l|c}
        Type & Examples \\
        \hline
        \large Browsers \normalfont & \texttt{chrome}, \texttt{firefox}, \texttt{konqueror} \\
        \large Text Editors \normalfont & \texttt{atom}, \texttt{code}, \texttt{kate}, \texttt{gedit}, \texttt{vim}, \texttt{emacs}, \texttt{nano} \\
        \large File Explorers \normalfont & \texttt{dolphin}, \texttt{nautilus} \\
        \large PDF Viewers \normalfont & \texttt{okular}, \texttt{evince} \\
\end{tabular}

\raggedright

We'd recommend exploring each of these applications in your own time, and find the ones you prefer. Some offer more functionality, others offer ease of use.

\section*{Tips}

You can clear the shell either by typing \texttt{clear}, or using the shortcut \texttt{Ctrl+L}.

You can navigate through your command history using the up and down arrow keys.

You can press tab to autocomplete the name of a file, if there is no ambiguity.

Generally, you should try to avoid creating directories with a space character in their names. 
This is because arguments passed to a command are separated by spaces.
If you do, you'll need to surround the name in quotes or put a backslash before every space.

If you're running Python, be careful to pick the correct version of it! 
\texttt{python} refers to Python 2, \texttt{python3} to Python 3.6, and we also have \texttt{python3.8}.

The method to exit the shell program you're currently using can vary depending on what it is.
Three of the most common ways to exit are \texttt{Ctrl+C}, \texttt{Ctrl+Z}, and \texttt{q}.

The same issue carries over to using the built-in command documentation. 
Although we used \texttt{man} in the lab, this won't always work - two alternatives are the \texttt{---help} flag and \texttt{info} command.

If you want to install an IDE with more features such as IntelliJ, PyCharm, or Eclipse, you can 
- but be mindful not to run out of storage space when doing so! 
You get 6GB in your first year, this is raised by 3GB for each subsequent year of your degree. 
To see how much you've got left, use the \texttt{quota} command.

Although it might seem pointless to use the terminal when you have a file explorer like Dolphin available, it doesn't offer as much functionality. Not only that, but over time you'll find that it might be faster to navigate using the terminal...

\newpage

\section*{Commands and Options}

(You can compose all single-letter command options: \texttt{-a} and \texttt{-l} can instead be \texttt{-al})

\Large \texttt{help} \normalsize \\
Built-in function help

\Large \texttt{ls <path>} \normalsize \\
List directory contents of the specified path

\qquad \texttt{-l} | Output in a detailed format \\
\qquad \texttt{-R} | Recursively list the contents of subdirectories \\
\qquad \texttt{-t} | Sort in order of when the file was last modified

\Large \texttt{cd <path>} \normalsize \\
Change directory

\Large \texttt{touch <filepath>} \normalsize \\
Make empty files

\Large \texttt{mkdir <filepath>} \normalsize \\
Make directories
\begin{itemize}
    \item \texttt{-p} | Create intermediate directories as required (if not used, full path must exist)
\end{itemize}

\Large \texttt{rm <filepath>} \normalsize \\
Remove directory entities
\begin{itemize}
    \item \texttt{-r} | Recursively attempt to remove each subdirectory and file.
    \item \texttt{-f} | Remove files without confirmation, if applicable
\end{itemize}


\Large \texttt{cp <source file> <target file>} \normalsize \\
Copy files from one location to another
\begin{itemize}
    \item \texttt{-r} | Can copy directory contents recursively.
    \item \texttt{-n} | Do not overwrite any existing files.
    \item \texttt{-u} | Only copy if the source file is newer than the destination file.
\end{itemize}

\Large \texttt{chmod <permissions> <filepath>} \normalsize \\
Modify the permissions of a file
\begin{itemize}
    \item \texttt{-R} | Recursively modifies permissions inside a subdirectory.
\end{itemize}



\end{document}